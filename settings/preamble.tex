\documentclass[12pt, a4paper]{report}
\usepackage[english]{babel}

% \usepackage[utf8]{inputenc}
% \usepackage[T1]{fontenc}

% \usepackage{polyglossia}
% \setdefaultlanguage{english}
% \setotherlanguage{german}
\usepackage[T1]{tipa}

\usepackage{fontspec}
% \setmainfont[Ligatures={TeX}]{Linux Libertine O}
% \setmainfont{Times New Roman}
\setmainfont{texgyretermes-regular} % best new times roman?
  [Extension      = .otf,
  BoldFont       = texgyretermes-bold,
  ItalicFont    = texgyretermes-italic,
  BoldItalicFont = texgyretermes-bolditalic
  ]
% \setmainfont{Tex Gyre Termes}


% \usepackage[]{newtxtext} %Times New Roman

\setmonofont[Scale=0.8]{Menlo}

% \usepackage[utf8]{inputenc}
\usepackage{csquotes}

%IPA
% \usepackage{tipa}
% \usepackage{tipx}




% Math
\usepackage{amsmath}
\usepackage{unicode-math}
\setmathfont{texgyretermes-math.otf}

% A package for  glosses (load after math!!)
\usepackage{gb4e}
\noautomath

% Control the width of captions of figures
\usepackage{caption}


% Quoting
\usepackage{quoting}

% Definition vom Zeilenabstand
\usepackage{setspace} % Zeilenabstand
\onehalfspacing %\doublespace or \singlespace

% Definition vom Seitenlayout
\setlength{\topmargin}{-1.2cm}
\setlength{\oddsidemargin}{0.5cm} 
\setlength{\evensidemargin}{0.5cm}

\setlength{\textheight}{24.5cm} 
\setlength{\textwidth}{15cm}

\setlength{\footskip}{1.2cm} 
\setlength{\footnotesep}{0.4cm}


% \usepackage{times} % Times New Roman

% Bibliography
\usepackage[style=authoryear]{biblatex}
\addbibresource{bibliography/bibliography.bib}
\addbibresource{bibliography/urls.bib}
\addbibresource{bibliography/word_align.bib}

% Graphics
\usepackage{graphicx}

\usepackage[svgnames]{xcolor}


% Create word alignment graphs
\usepackage{tikz}
\usetikzlibrary{tikzmark}
\usetikzlibrary{shapes.geometric, arrows}
\tikzstyle{startstop} = [rectangle, rounded corners, minimum width=3cm, minimum height=1cm,text centered, draw=black, fill=blue!30]
\tikzstyle{arrow} = [thick,->,>=stealth]

\tikzset{every tikzmarknode/.style={align=center, inner sep = 1pt,execute at end node={\vphantom{bg}}}}

% Attempts to plot word alignments
% \newcommand*{\hnode}[1]{\node[outer sep=0pt,anchor=base] (#1) {#1};} % create a labelled node

% \usepackage{pstricks}% http://www.tug.org/PSTricks/main.cgi/
% \usepackage{pst-node}
% \usepackage{calc}% For width calculations
% \newcommand*{\Tword}[1]{\rnode{#1}{\raisebox{1ex}{\smash{#1}}}}% \Tword{<top>}
% \newcommand*{\Bword}[1]{\rnode{#1}{\raisebox{-2ex}{\smash{#1}}}}% \Bword{<bottom>}
% \newcommand{\TtoB}[3][]{% \TtoB{<top>}{<bottom}
%   \ncdiag[arm=1em,angleA=-90,angleB=90,linestyle=solid,linecolor=black,linewidth=0.5pt,#1]{#2}{#3}}%

% \renewcommand{\sfdefault}{phv}
% \renewcommand{\rmdefault}{ptm}
% \renewcommand{\ttdefault}{pcr}

% colors

\usepackage{listings}
\definecolor{background}{HTML}{EEEEEE}
% json listings
\lstdefinelanguage{json}{
    basicstyle=\fontsize{10}{10}\normalfont\ttfamily,
    numbers=left,
    numberstyle=\scriptsize,
    stepnumber=1,
    numbersep=8pt,
    showstringspaces=false,
    breaklines=true,
    backgroundcolor=\color{background},
    frame=lines}

\lstdefinelanguage{txt}{
    basicstyle=\fontsize{10}{10}\normalfont\ttfamily,
    showtabs=true,
    tab=\rightarrowfill,
    numbers=left,
    numberstyle=\scriptsize,
    stepnumber=1,
    numbersep=8pt,
    showstringspaces=false,
    breaklines=true,
    backgroundcolor=\color{background},
    frame=lines}


\usepackage{numprint}
\npthousandsep{,}

\usepackage{dirtree}

% Table stuff
\usepackage{multirow}
\usepackage{tablefootnote}
\usepackage{booktabs} % schöne Tabellen

\usepackage{subcaption} % for subfigures

\usepackage{nameref}
\newcommand{\refnameref}[1]{\ref{#1}:~\nameref{#1}}
% \newcommand{\defref}[1]{Definition~\ref{#1}}
\usepackage[colorlinks=true,linkcolor=black,anchorcolor=black,citecolor=DarkBlue,filecolor=black,menucolor=black,runcolor=black,urlcolor=DarkBlue]{hyperref}

% Acronyms
% \usepackage{acro}
% \DeclareAcronym{aer}{
  short=AER,
  long=Average Error Rate,
}
\usepackage[acronym]{glossaries}
% \makeglossaries
% \makeglossaries
\makenoidxglossaries

\newglossaryentry{graubuenden}
{
    name=Graubünden,
    description={The Canton of Grisons}
}

\newglossaryentry{recall}
{
    name=recall,
    description={Percent of missing positives. 
    Calculated as true positives divided by all positives (true positives plus false negatives) $\frac{\text{TP}}{\text{TP}+\text{FP}}$}
}


\newacronym{aer}{AER}{Average Error Rate}
% \makenoidxglossaries
\hyphenation{SimAlign} % no hyphenation for SimAlign


% No new page after abstract
% \usepackage{etoolbox} % provides \patchcmd
% \patchcmd{\endabstract}{\null}{}{}{} % replaces \null with third argument (empty)

