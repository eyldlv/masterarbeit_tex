\documentclass[11pt, a4paper]{report}
\usepackage[english]{babel}

%\usepackage[utf8]{inputenc}
% \usepackage[T1]{fontenc}
\usepackage{fontspec}
% \setmainfont[Ligatures={TeX}]{Linux Libertine O}

% \usepackage{times}
\usepackage[]{newtxtext} %Times New Roman
\setmonofont[Scale=0.8]{Menlo}
\usepackage[T1]{tipa}
\usepackage{csquotes}

%IPA
% \usepackage{tipa}
% \usepackage{tipx}

% Create glosses
\usepackage{gb4e}
\noautomath

\usepackage{amsmath}


% Schönere Tabellen
\usepackage{booktabs}

% Definition vom Zeilenabstand
\usepackage{setspace} % Zeilenabstand
% \onehalfspacing %\doublespace or \singlespace

% Definition vom Seitenlayout
\setlength{\topmargin}{-1.2cm}
\setlength{\oddsidemargin}{0.5cm} 
\setlength{\evensidemargin}{0.5cm}

\setlength{\textheight}{24.5cm} 
\setlength{\textwidth}{15cm}

\setlength{\footskip}{1.2cm} 
\setlength{\footnotesep}{0.4cm}


% \usepackage{times} % Times New Roman

\usepackage[style=authoryear]{biblatex}
\addbibresource{bibliography/bibliography.bib}
\addbibresource{bibliography/urls.bib}



% Create word alignment graphs
\usepackage{tikz}
\usetikzlibrary{tikzmark}
\tikzset{every tikzmarknode/.style={inner sep = 1pt,execute at end node={\vphantom{bg}}}}

% Attempts to plot word alignments
% \newcommand*{\hnode}[1]{\node[outer sep=0pt,anchor=base] (#1) {#1};} % create a labelled node

% \usepackage{pstricks}% http://www.tug.org/PSTricks/main.cgi/
% \usepackage{pst-node}
% \usepackage{calc}% For width calculations
% \newcommand*{\Tword}[1]{\rnode{#1}{\raisebox{1ex}{\smash{#1}}}}% \Tword{<top>}
% \newcommand*{\Bword}[1]{\rnode{#1}{\raisebox{-2ex}{\smash{#1}}}}% \Bword{<bottom>}
% \newcommand{\TtoB}[3][]{% \TtoB{<top>}{<bottom}
%   \ncdiag[arm=1em,angleA=-90,angleB=90,linestyle=solid,linecolor=black,linewidth=0.5pt,#1]{#2}{#3}}%

% \renewcommand{\sfdefault}{phv}
% \renewcommand{\rmdefault}{ptm}
% \renewcommand{\ttdefault}{pcr}

% colors
\usepackage{xcolor}
\usepackage{listings}
\definecolor{background}{HTML}{EEEEEE}
% json listings
\lstdefinelanguage{json}{
    basicstyle=\fontsize{10}{10}\normalfont\ttfamily,
    numbers=left,
    numberstyle=\scriptsize,
    stepnumber=1,
    numbersep=8pt,
    showstringspaces=false,
    breaklines=true,
    backgroundcolor=\color{background},
    frame=lines}

\usepackage{dirtree}

\usepackage{hyperref}


