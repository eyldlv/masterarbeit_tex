\chapter{Concluding Words}\label{chap:summary}

\section{Goals}
The goals of this work was twofold: 
\begin{itemize}
	\item Enlarge the amount of digital resources that is available containing the Romansh language;
	\item Evaluate a novel similarity based word alignment method which uses word embeddings on the language pair German-Romansh.
\end{itemize}


\section{Corpus Compliation}
In order to achieve both goals, I first had to collect data. 
I chose to collect the press releases published by the \emph{\Gls{standeskanzlei}} of the canton of \Gls{graubuenden} since 1997 until today. 
These press releases have been released in the three official languages of the canton: German, Italian and Romansh. 
I aligned the press releases (henceforth \emph{documents}) using URL matching when possible, or reverted to a simple heuristic (three releases from the same day in three different languages are mutual translations).
The documents (aligned and not aligned), are saved both as JSON files and in a SQLite database, which allows for fast and simple queries.

I then proceeded to align the sentences using hunalign \autocite{hunalign}, a fast length- and dictionary-based method for aligning sentences. 
After filtering noise (duplicates and misalignments), as well as sentences containing only phone numbers, URLS or email addresses, I was able to produce around 80,000 sentence pairs for each language combination (German-Romansh, German-Italian, Romansh-Italian).

I will be glad to provide make the corpus that I collected, as well as the aligned sentence pairs, to other students for further research and experimentation{\interfootnotelinepenalty10000\footnote{In case you would like to use this corpus, please consult the copyright notice on \url{https://www.gr.ch/de/Seiten/Impressum.aspx}} before publicly releasing it of parts thereof.}. 

\section{Gold Standard}
In order to evaluate word alignment systems, a gold standard is needed. 
In the context of word alignment, a gold standard is a collection of sentence pairs annotated for word correspondences. 
Since there is no gold standard for German-Romansh, I annotated word correspondences in 600 sentences (see Chapter~\ref{chap:gold-standard}). I will  gladly provide my annotations to other students for further experiments and research. 

\section{Evaluation}
I compared the performance of statistical word alignment methods---fast\_align \autocite{dyer-etal-2013-simple} and eflomal \autocite{Ostling2016efmaral}---with the novel similarity and embeddings based method SimAlign \autocite{jalili-sabet-etal-2020-simalign}. 
SimAlign's performance is on par with fast\_align, but was outperformed by eflomal. 
This still shows that SimAlign is a viable method for computing word alignments for German-Romansh. 
Considering the fact that the multilingual embeddings used by SimAlign (mBERT) do not contain embeddings for Romansh (a.k.a.~zero-shot setting), I believe these results are very promising.

\section{Future}
The corpus I collected might be used by future students in a variety of ways. 
One way that comes to mind is training a neural machine translation model using the $\sim80,000$ sentence pairs I extracted and testing a variety of methods for enriching using monolingual data, such as back-translation (an automatic translation of the monolingual target text into the source language) \autocite{sennrich-etal-2016-improving}. 
See also \cite{https://doi.org/10.48550/arxiv.2107.04239}.

Another possibility would be to fine-tune or extend mBERT with Romansh data. 
Enlarging the vocabulary of mBERT to accomodate an unseen language and then continue training the model on this language was shown to significantly improve performance in an NER task for that lanauge compared to a zero-shot setting \autocite{wang-etal-2020-extending}. 

It would also be desirable that a future student would repeat my annotations of the 600 sentences as a second annotator. 
This would make the gold standard more sensible, reliable and acceptable, and would introduce a set of Possible alignments to it (see Section~\ref{sec:gold-flaws}).




