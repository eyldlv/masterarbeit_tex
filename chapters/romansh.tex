\chapter{Romansh}

In this chapter, I will provide a short context about Romansh, the language that builds a third of the resulting corpus, but conceptually the main motivation for this work.

\section{Raeto-Romance}
In 1873, an Italian linguist by the name of Grziadio Ascoli pointed out to a shared number of chracterizing phenomena in a number of Romance dialects spoken in parts of Switzerland and Italian (but without a geographical continuum) and named this group of dialects \enquote{Ladino}. 

Since 1883, influenced by Theodor Gartner's publication \emph{Raetoromanishce Grammatik} on this group of dialects, this name (German \emph{Rätoromanisch}, English \enquote{Raeto-Romance}) became associated with them. 
Raeto-Romance is spoken in three separated areas and is made up of three higher dialects: Romansh, spoken in parts of the Swiss canton of Graubünden, Ladin, spoken in the Dolomotic Alps in northern Italy (Südtirol), and Friualian, spoken around the drainage basin of the Tagliamento river, between Venice and Trieste \autocite[1]{haiman1992}.

There have been long discussions in Romance linguistics about whether Raeto-Romance can be seen as a unity of dialects, or whether such a unity is merely a linguistic construct, lacking a sociolinguistical-historical basis. 
This dispute, referred to as the \emph{questione ladina} \enquote{the Ladin question} \autocite{liver1999}. 

Ascoli, the grounder of the idea of one Raeto-Romance unity, made his classifications at a time where language researchers were fascinated by the regularity of sound changes and common historical sound changes were used to group languages and dialects together. He therefore based his grouping together of these dialects on sound changes common to all three areas. 
His followers propagate a narrative according to which the three dialects once occupied one geographical area, but were seperated by the Germanic intruders \autocite[174]{bossong2008}.

An opposing group of reserchers believes that the three Raeto-Romance dialects show decisive  features common with their respective Italian neighbors. 
They should therfore be classified as an Italian dialect and part of the Italian dialect continuum \autocite[174]{bossong2008}.

This question is not of importance to this thesis and will not bother us in the course of it. 
It is, however, important to remember that names and definitions posed by academics do not always correspond to the feelings of the speakers and their own sense of identity. 
In this case, speakers of these dialects do not feel as though they all belong to some greater unity \autocite[175]{bossong2008}.

\section{Romansh}
The term Roamansh is a collective name referring to the Raeto-Romance dialects spoken in Switzerland and is recognized as a single language. 
There are five different dialects (Surselvan, Sutselvam, Suermeran, Puter, Vallader), with each having normative grammars and distinct orthographic norms (motivated by the Reformation, for translating the Bible and other religious texts) \autocites[1]{haiman1992}[178]{bossong2008}.

Romansh was officially acknowledged as a fourth official language in Switzerland (besides German, French and Italian) in a federal referendum that took place in 1938, in the eve of the Second World War, with a whopping majority of 92\% Yes votes. 
It has been shown that this referendum played in the hands of the raeto-romanians in Graubünden to promote their nationalistic political postulate, but was also instrumentalised by the Swiss federal government to counter Mussolini's pretensions to enquote{Italian} terrotires in Switzerland (referred to as the Italian irredentism) \autocite{valaer2012}.

They are currently spoken by around 40,000 people \autocite{bundesamt2020}. 
This number has been diminishing constantly -- 30 years ago there were 50,000 speakers \autocite{haiman1992}. 
There is however hardly a single person who only speaks Romansh. 
In Switzerland, as in the other regions of Raeto-Romance, there is always a \enquote{prestige} language surrounding Raeto-Romance, in which Raeto-Romance speakers are fluent in \autocite[3]{haiman1992}. 

\section{Rumantsch Grischun}
\subsection{Lia Rumantscha}
In the last hundred years there have been a Raeto-Romance revival. 
In Switzerland, a major force in this movement has been the founding of the Lia Rumantscha \enquote{The Romansh League} in 1919, which was also a counter-force to the Italian irredentism \footnote{The nationalistic claim of lands inhabited by persons who the Italian nationalists saw as ethnic Italians}. 
It is an umbrella organisation devoted to promoting and perserving the Raeto-Romance language and culture. Its goals include creating and promoting a common language awareness and identity amongst the Raeto-Romans, be responsible for developing a language standard and language renovation, and generally represent the interests of the Romansh and its speakers, in Graubünden and in the Swiss diaspora \autocite{dazzi2012}.

\subsection{Rumantsch Grischun}
The endeavours of the Lia Rumantscha in the field of language planning and standardization led to the official launching of a pan-Romansh language -- \enquote{Rumantsch Grischun} \autocite[5]{haiman1992}. 
Its goal was not to replace the local dialects, but be available for persons, institutions, government agencies, companies etc., that would like to use Romansh but require a language variant that would be interrgional and intelligible by speakers of all dialects. The main motivation for planning an interregional standard was the failure of Romansh to establish itself as a fourth national language, despite the great willingness of the people, due to the lack of a written standard. 
The existence of a written standard would make Romansch to be better respected and incorporated in the canton of Graubünden, as well as on a federal level, and would elevate its prestige in the eyes of its speakers \autocite{schmid1982}.

\subsection{Properties}
Rumantsch Grischun was suggested in 1982 by the Zurich born Romance linguist Heinrich Schmid. 
It was, however, not the first attempt to harmonize the Romansh dialects. 
In the 19th century, a high school teacher named Gion Antoni Bühler, made failed attempts to propagate for a \emph{Romonsch fusionau}; in the 1960's a Swiss author from the canton of Graubünden, Leza Uffer, suggested \emph{Interrumantsch}, which was mainly based on the Surmiran dialect, but failed similarly \cite[39]{liver1999}.

Rumantsch Grischun's success has been hypothesized to be mainly due to the favorable timing -- the socioeconomical situation at the time as well as a change in the approach of many Raeto-Romans to their own language; but also due to the fact that Rumantsch Grischun, contrary to the former suggestions, is more consistent and balanced between the dialects \autocite[69]{liver1999}.

Without going to much into details, Rumantsch Grischun favors the greatest common denominator. 
It takes the word forms common to the three most important written dialects (Sursilvan, Surmiran and Vallader). For instance, in all three dialect the word for \enquote{key} is \emph{clav}, hence, this is also the Rumantsch Grischun word for \enquote{key}.
In case the dialects do not agree, it includes the word form common to the most dialects in a sort of \enquote{majority vote}. 
That way, it never prefers one dialect over the other throughout. 
Clarity and transperence also play a major role. 
This means that forms which exhibit stem alternations, for instance between singular and plural, are abandoned in favor of the simpler, more constant form.
Also phenomenons that are specific to just one dialect are left outside, such as the rounded front-vowels [y] and [ø] typical of the dialects of the Engadine or the closing diphthong [ɪw]\footnote{The diphthong starts with an open vowel [ɪ] and ends with a closed vowel [w]}, which is unique to Sursilvan \autocite[70]{liver1999}.

This new language fulfills the requirements of its authors: it can be read and understood by any Raeto-Roman without having to elaborately learn it and the differences to the specific dialects are minimal \autocite[72]{liver1999}. 


\begin{table}
\centering
\begin{tabular}{lllll}
\toprule
Sursilvan & Surmiran & Vallader & Rumantsch Grischun & Principle \\
\midrule
\textbf{clav}	  &  \textbf{clav}    &  \textbf{clav}    & \textbf{clav}  \enquote{key}  & Greatest common denominator \\
\textbf{tschiel}       &   \textbf{tschiel}   & tschel      & \textbf{tschiel} \enquote{sky} & Majority vote \\
siat 	 & \textbf{set}      & \textbf{set}      & \textbf{set} \enquote{seven} & " \\
\textbf{cor} & \textbf{cor} & cour & \textbf{cor} \enquote{heart} & " \\
vendiu & vendia & vendü & \textbf{vendi} \enquote{bought} & Favor simplicity \\
sg./pl. \emph{iert}/\emph{orts} & \textbf{iert/ierts} & üert/üerts & iert/ierts \enquote{garden} & " \\
\bottomrule

\end{tabular}
\caption{Examples for choosing the forms for Rumansch Grischun, based on \cite[70-71]{liver1999}}
\label{tab:rg-examples}
\end{table}


\subsection{Today}
Since then, Romansh and the people promoting it have had notable success achieving their goals. 
In 1999, Romansh became a \enquote{partially official language} (\emph{Teilamtssprache}) of the Swiss confederation. 
In 2003, it was recognized in the cantonal constitution of Graubünden as an equal cantonal language, and the protection of the traditional language regions was guaranteed.
Nowadays, Romansh is in use in many domains, not only in the public administration, but also in the business cycle. 
Rumantsch Grischun has become one of the most ambitious endeavours in the history of Romansh. 
Many writings and works were written in Rumantsch Grischun. 
People learn to read and write in Rumantsch Grischung and in some schools, classes are held in Rumantsch Grischun. 
The extent of radio and television in Romansh has been growing. 
There is a radio station broadcasting 24/7, television programs broadcast in all public channels of the Swiss Broadcating Corporation (SSG SSR), as well as  internet portals, e.g., \url{https://www.rtr.ch/}. All of this were not possible if it weren't for the political \enquote{upgrade} that was aspired for by the Romansh language movement \autocite{cathomas2012}.

The canton of Graubünden has been releasing most or all of its press releases, which build up this corpus, in three languages: German, Italian and Romansh using the Rumantsch Grischun standard.




