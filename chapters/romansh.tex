\chapter{Romansh}

In this chapter, I will provide a short context about Romansh.

In 1873, an Italian linguist by the name of Grziadio Ascoli pointed out to a shared number of chracterizing phenomena in a number of Romance dialects spoken and named this group of dialects \enquote{Ladino}. 
Since 1883, influenced by Theodor Gartner's publication \emph{Raetoromanishce Grammatik} on this group of dialects, this name (German \emph{Rätoromanisch}, English \enquote{Raeto-Romance}) became associated with them. 
Raeto-Romance is spoken in three separated areas: in parts of the Swiss canton of Graubünden, in the Dolomotic Alps in northern Italy (Südtirol) and around the drainage basin of the Tagliamento river, between Venice and Trieste \autocite[1]{haiman1992}.

There have been long discussions in Romance linguistics about whether Raeto-Romance can be seen as a unity of dialects, or whether such a unity is merely a linguistic construct, lacking a sociolinguistical-historical basis. 
This dispute, referred to as the \emph{questione ladina} \enquote{the Ladin question} \autocite{liver1999}. 
This question is not of importance to this thesis and will not bother us in the course of it. 
It is, however, important to remember that names and definitions posed by academics do not always correspond to the feelings of the speakers and their own sense of identity.

The term Roamansh is a collective name referring to the Raeto-Romance dialects spoken in Switzerland and is recognized as a single language. 
There are five different dialects (Surselvan, Sutselvam, Suermeran, Puter, Vallader), with normative grammars and distinct orthographies (motivated by the Reformation, for translating the Bible and other religious texts) \autocite[1]{haiman1992}.

Romansh was officially acknowledged as a fourth official language in Switzerland (besides German, French and Italian) in a federal referendum that took place in 1938, in the eve of the Second World War, with a whopping majority of 92\% Yes votes. 
It has been shown that this referendum played in the hands of the raeto-romanians in Graubünden to promote their nationalistic political postulate, but was also instrumentalised by the Swiss federal government to counter Mussolini's pretensions to enquote{Italian} terrotires in Switzerland (referred to as the Italian irredentism) \autocite{valaer2012}.

They are currently spoken by around 40,000 people \autocite{bundesamt2020}. 
This number has been diminishing constantly -- 30 years ago there were 50,000 speakers \autocite{haiman1992}. 
There is however hardly a single person who only speaks Romansh. 
In Switzerland, as in the other regions of Raeto-Romance, there is always a \enquote{prestige} language surrounding Raeto-Romance, in which its speakers are fluent in \autocite[3]{haiman1992}. 

In the last hundred years there have been a Raeto-Romance revival. 
In Switzerland, a major force in this movement is the founding of the Lia Rumantscha \enquote{The Romansh League} in 1919, which was also a counterforce to the Italian irredentism \footnote{The nationalistic claim of lands inhabited by what the Italian nationalists saw as ethnic Italians}. 
It is an umbrella organisation devoted to promoting and perserving the Raeto-Romance language and culture. 

