\chapter{Sentence Alignment}

\section{Introduction}

The corpus presented in chapter~\ref{chap:compiling} is a raw parallel corpus, that is it is a corpus of aligned documents without any further processing. 
In order to use the corpus for tasks such as training a machine translation model, another processing step is needed: sentence alignment \autocite[55]{koehn2009}.

Formally, the task can be described as follows: We have a list of sentences in language \(e\), \(e_1,...e_{n_e}\) and a list of sentences in language \(f\), \(f_1,...,f_{n_f}\). 
(Note that the number of sentences in each language is not necessarily identical.) 
A sentence alignment \(S\) consists of a list of sentence pairs \(s_1, ..., s_n\), such that each sentence pair \(s_i\) is a pair of sets:

\[
	s_i = ( \{ e_{\text{start-e}(i)},... , e_{\text{end-e}(i)}\}, \{f_{\text{start-f}(i)},... , f_{\text{end-f}(i)}\} )
\]
\autocite[56]{koehn2009}

This means each set in the pair of set can consist of one or more sentences. 
The numer of sentences in each set is referred to as alignment type. 
A 1-1 alignment is an alignment where exactly one sentence of language \(e\)
is aligned to exactly one sentence of language \(f\). 
In a 1-2 alignemnt, one sentence in lanauge \(e\) is a aligned to two sentences in langauge \(f\). 
There are also 0-1 alignments, in which a sentence of language \(f\) is not aligned to anything of language \(e\). 
Sentences may not be left out and each sentence may only occur in one sentence pair \autocite[57]{koehn2009}. 


\section{Methods}
One early method for sentence alignment is the one described in \cite{gale-church-1991-program} which is \enquote{based on a simple satistical model of character lengths} \autocite{gale-church-1991-program}. The method arose out of the need to design a faster, computationally more efficient algorithm\footnotemark.

\footnotetext{With the algorithms that existed up to that time, it took 10 days to extract 3 million sentence pairs, 12,500 sentences per hour.}

The method uses the fact that longer sentences in language \(e\) are usually translated into longer sentences in language \(f\) and vice-versa -- shorter sentences correspond to shorter sentences.

The method combines a distance measure based on the lengths of the sentence with a prior probility of the alignment type (1-1, 1-0 or 0-1, 2-1 or 1-2, 2-2) to a probabilistic score. 
It assigns this score to possible sentence pairs in a dynamic programming framework to find the best (most probable) pairs. 
A program based on this method was tested against a human-made alignment on two pairs of languages: English-German and English-French. 
The program made a total of 55 errors out of a total of 1316 alignments (4.2\%). 
By taking the best scoring 80\% of the alignments, the error rate could be reduced to 0.7\%

The method was also much faster than the algorithms that existed up to that time. 
It took 20 hours to extract around 890,000 sentence pairs, around 44,500 sentence pairs per hour, around 3.5 times faster than former algorithms.


\section{Statistical methods}


\section{Recent methods}