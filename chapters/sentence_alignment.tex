\chapter{Sentence Alignment}\label{chap:sentence-alignment}
% \begin{refsection}
\section{Introduction}

The corpus presented in chapter~\ref{chap:compiling} is a raw parallel corpus, that is, it is a corpus of aligned documents without any further processing. 
In order to use the corpus for tasks such as training a machine translation model, another processing step is needed: sentence alignment \autocite[55]{koehn2009}.

A bilingual, sentence-aligned corpus can be useful for a variety of tasks. 
Bilingual corpora are  probably mostly used for training a machine translation model \autocites{gale-church-1991-program,moore2002fast,chen-1993-aligning}, but they can also be used for building translation memories \autocite{sennrich-volk-2011-iterative} or a for bilingual concordance systems, with the purpose of allowing a user to find out how a given sentence is translated \autocites{moore2002fast,gale-church-1991-program}, e.g., multilingwis\footnote{\url{https://pub.cl.uzh.ch/projects/sparcling/multilingwis2.demo/}} \autocite{multilingwis}.

\subsection{Formal definition}
Formally, the task of sentence alignment can be described as follows: We have a list of sentences in language \(e\), \(e_1,...e_{n_e}\) and a list of sentences in language \(f\), \(f_1,...,f_{n_f}\). 
(Note that \(n_e\) the number of sentences in language $e$, is not necessarily identical to \(n_f\) the number of sentences in language \(f\).) 
A sentence alignment \(S\) consists of a list of sentence pairs \(s_1, ..., s_n\), such that each sentence pair \(s_i\) is a pair of sets:

\[
	s_i = ( \{ e_{\text{start-e}(i)},... , e_{\text{end-e}(i)}\}, \{f_{\text{start-f}(i)},... , f_{\text{end-f}(i)}\} )
\]
\autocite[56]{koehn2009}

This means that each set in this pair of sets can consist of one or more sentences. 
The number of sentences in each set is referred to as \emph{alignment type}. 
A 1-to-1 alignment is an alignment where exactly one sentence of language \(e\)
is aligned to exactly one sentence of language \(f\). 
In a 1-to-2 alignment, one sentence in language \(e\) is a aligned to two sentences in language \(f\). 
There are also 0-to-1 alignments, in which a sentence of language \(f\) is not aligned to anything of language \(e\). 
Sentences may not be left out and each sentence may only occur in one sentence pair \autocite[57]{koehn2009}. 


\section{Method Overview}
\label{sec:overview_senalign}
Traditionally, there are three main approaches for solving the problem of sentence alignment:  length-based, dictionary- or translation-based and partial similarity-based \autocite{hunalign}. 

\subsection{Length-Based}
One early method for sentence alignment is %the one described in \cite{gale-church-1991-program} which is 
\enquote{based on a simple statistical model of character lengths} \autocite{gale-church-1991-program}. The method, dubbed since as the \enquote{Gale \& Church method/algorithm}, arose out of the need to design a faster, computationally more efficient algorithm than the ones that existed at the time\footnotemark.

\footnotetext{With the algorithms that existed up to that time, it took 10 days to extract 3 million sentence pairs, 12,500 sentences per hour.}

The Gale \& Church method is based on the assumption that longer sentences in language \(e\) are usually translated into longer sentences in language \(f\) and vice-versa---shorter sentences in one language correspond to shorter sentences in the other language.

The method combines a distance measure based on the lengths of the sentence with a prior probability of the alignment type (1-to-1; 1-to-0 or 0-to-1; 2-to-1 or 1-to-2; 2-to-2) to a probabilistic score. 
It assigns this score to possible sentence pairs in a dynamic programming framework to find the best (most probable) pairs \autocite[57]{koehn2009}. 

A program based on this method was tested against a human-made alignment on two pairs of languages: English-German and English-French. 
The program made a total of 55 errors out of a total of \numprint{1316} alignments (4.2\%). 
By taking the best-scoring 80\% of the alignments, the error rate was able to be reduced to 0.7\%.
The method was also much faster than the algorithms that existed up to that time: 
It took 20 hours to extract around \numprint{890000} sentence pairs, around \numprint{44500} sentence pairs per hour, which is about 3.5 times faster than previous algorithms \autocite{gale-church-1991-program}.

\subsection{Partial Similarity-Based}
Another method is similarity-based such as the one presented in \cite{simard-plamondon-1996-bilingual}. 
Here, alignment follows two steps (or passes). 
In the first step, \emph{isolated cognates} are used to mark sort of \emph{anchors} in the texts. 
The term \enquote{cognate} refers here to two word-forms in different languages, whose  first four characters are identical. 
Isolated cognates are cognates with no resembling word forms within a context window.
It follows the assumption that two isolated cognates of different languages are parts of segments that are mutual translations and should be aligned with each other. 
These cognates are used as anchors, and the process is repeated recursively between the anchors, in order to find further isolated cognates within these boundaries, until no more anchor points can be found.

In an intermediate step, segmentation into sentence boundaries takes place and the search space is determined. 
In other words, based on the anchors found in the first step, it is determined which sentences could be aligned with each other. 
Only sentence-pairs that are within the same search space boundaries are alignment candidates.

In the second step, the final alignment takes place. 
Theoretically, any sentence alignment program that can operate within the restricted search space defined in the previous steps can take over the job. 
In \cite{simard-plamondon-1996-bilingual}, the authors use a statistical lexical translation model (commonly known as IBM Model 1, see Section~\ref{sec:ibm-model-1}), to measure how probable it is to observe one sentence given another sentence, and so find the sentences that are most likely mutual translations.

\subsection{Translation-Based}

Another possibility for aligning sentences is translation-based. 
Here, the alignment algorithm constructs a statistical word-to-word translation model of the corpus. 
It then finds the sentence alignment that maximizes the probability of generating the corpus with this translation model. 
In other words, it aligns sentences that are most likely translations of each other, given the translation model \autocite{chen-1993-aligning}. 

\subsection{Hybrid models}
\label{sec:hybrid-models}
There are also hybrid sentence-alignment methods, combining several methods.


\cite{moore2002fast} presents a method in which sentence lengths are combined with word correspondences to find the best alignments. 
It works in three steps:
First, sentences are aligned using a sentence-length-based model. 
Then, the sentence pairs with the highest probability, i.e., those that are most likely real correspondences of each other, are used to train a translation model. 
The translation model is then used to augment the initial alignment, so that the result is length- and translation-based \autocite{moore2002fast}.

Another hybrid method was presented by \cite{hunalign}. 
It combines a dictionary- and a length-based method.
Here, a sort of a dummy translation of the source text is produced using a translation dictionary which is supplied to the program\footnote{Note that this is not a real restriction. See Section~\ref{sec:aligning-language-pairs}}. 
The program then simply converts each token into its corresponding dictionary translation.
After the dummy translation has been created, a similarity score is computed for each sentence pair.
The similarity score consists of two components: a score based the number of shared words in the sentence pair (token-based) and  a score based on the ratio of character counts between sentences (length-based). 
The program treats paragraph boundaries (special \texttt{<p>} tokens) as sentences with special scoring. 
This similarity score of a paragraph-boundary  and a real sentence is always minus infinity, which makes sure they never align. This way, paragraph boundaries always align with themselves and can be used as anchors to keep  paragraphs mutually aligned \autocite{hunalign}.

\subsection{Summary}
All the methods presented here perform very well on clean, well-structured data in similar languages. Already the Gale \& Church algorithm from 1993 achieved a precision of 98\% on the Canadian Hansards\footnote{Transcriptions of parliamentary debates which exist in English and in French}, which \citeauthor{gale-church-1991-program} acknowledge are easy to align. 
What seems to have led researchers to develop better sentence alignment algorithms are speed \autocites{chen-1993-aligning,hunalign} and better performance on noisy data (such as 1-to-many alignments and misrecognized paragraph boundaries \autocite{sennrich-volk-2010-mt}). 

While speed might be considered a mundane issue, when working with noisy data or with a large amount of data, several alignment runs might be required until misalignments can be detected. 
When the alignment process takes less time,  texts that are less suitable for alignment (mixed order of chapters, different prefaces, etc.) 
can be filtered out earlier, and pre-processing steps such as tokenization and sentence segmentation, which may also influence the alignment quality, can be tested. 
Tweaking and fine-tuning the model parameters may also require several runs \autocite{hunalign}. 

In other words, it may take several attempts until unsuitable texts can be filtered out, the best pre-processing steps are identified, and the best model parameters are found. 
An algorithm which performs faster has a clear advantage in such cases.

% In other words, sentence alignment for a big corpus often requires several passes or runs until misalignments due to less suitable texts or faulty tokenization and sentence segmentation are revealed. An algorithm that performs faster has an obvious advantage here.

\section{More Recent methods}
While the statistics- and length-based  methods described in section~\ref{sec:overview_senalign} date back to the 1990's, more recently other methods were suggested.

\subsection{Bleualign}

One of these methods was presented in \cite{sennrich-volk-2010-mt} and has been dubbed since as Bleualign. 
It arose as a method addressing  the problem of aligning less \enquote{easily} alignable corpora. 
Sentence alignment methods up to that time perform excellent on well-structured corpora with a high language similarity such as the Canadian Hansards  or the Europarl\footnote{Parliamentary proceedings of the EU Parliament}  which are considered easy to align because they are well-structured---they provide markup information to identify speakers which is useful for creating anchor points and the subsequent alignments \autocites{simard-plamondon-1996-bilingual,sennrich-volk-2011-iterative}. 
However, when aligning pairs of languages which are fundamentally different and/or of less structured texts, the alignment task becomes more difficult \autocite{sennrich-volk-2010-mt}.

Bleualign uses BLEU as a similarity score to find sentence alignments. 
BLEU, which stands for Bilingual Evaluation Understudy, is a popular automatic metric for evaluating machine translation models. It measures the similarity between two sentences by considering matches of several n-grams\footnote{Sequences of tokens of length \(n\)}\,\footnote{Usually scores are combined for n-grams of order 1 to 4.}. 
The higher the BLEU score, the higher the similarity between two sentences \autocite[226]{koehn2009}.

Although BLEU has been criticized as a measure of translation quality, BLEU scores can be used for deciding whether two sentences are mutual translations:  
The higher the BLEU score, the more likely it is that two sentences are mutual translations. 
BLEU scores for two unrelated sentences is usually 0. Instead of aligning sentences of the source and the target language with each other, Bleualign aligns a machine translated version of the target side of the corpus with the source side in order to find the most reliable alignments \autocite{sennrich-volk-2010-mt}. 

However, this approach requires an already existing machine translation system with reasonable performance. 
This problem was addressed in \cite{sennrich-volk-2011-iterative} by suggesting an iterative method for alignment, combining length-based and BLEU score-based methods, which doesn't require an already existing machine translation system. 
In the first iteration, sentences are aligned using an implementation of the Gale \& Church algorithm, then a \acrfull{smt} system is trained on the sentence-aligned corpus. 
In the following iterations, the corpus (target side) is machine-translated using the \acrshort{smt} system trained in the last iteration and is then aligned to the source side using Bleualign. 
Then, a new \acrshort{smt} system is trained using the current alignments.

\cite{sennrich-volk-2011-iterative} do not recommend this iterative sentence alignment procedure for all purposes. 
It should be used mainly where conventional sentence alignment algorithms such as Gale \& Church have lower accuracy or where language-specific resources such as dictionaries (needed for hunalign \autocite{hunalign}) or machine translation systems are unavailable or lacking in quality.

\subsection{Vecalign}
The desire for  sentence alignment of even higher quality rose with the insight that, while misaligned sentences have small effect on \acrshort{smt} performance, they do have a crucial effect on \acrfull{nmt} systems. 
This is especially true in scenarios with less data for low-resource \acrshort{nmt} \autocite{thompson-koehn-2019-vecalign}. 

Vecalign uses a novel method which is based on the similarity of bilingual sentence embeddings. 
Sentence embeddings are, in a manner similar to word embeddings (see Section~\ref{sec:word-embeddings}), vector representation of sentences that are learned by and can be extracted from a neural language model. 
This vector representation is said to represent the meaning of a sentence. 
The sentence embeddings are obtained from a language model that was trained on multiple languages, thus, the embeddings for all languages share the same vector space.
This means that the embeddings are indifferent to the specific input language: They are language agnostic.
If two sentences, regardless of their language, are similar, their vector representations will lie close to each other in the vector space. 
A function that is most often used for measuring vector similarity is the cosine similarity (see Section~\ref{sec:word-similarity}.
In this manner, similar sentences in different languages can be identified and aligned \autocite{artexte-schwenk-2019-laser}.


\section{Sentence Alignment Pipeline}

I shall now describe the steps I took for extracting sentence pairs out of  the corpus I compiled in section~\ref{chap:compiling}.

\subsection{Tool of choice}
\label{subsec:tool}
My tool of choice was hunalign \autocite{hunalign}. 
It is presented as a software package on GitHub, it is free to use and contrary to the Microsoft program presented by \cite{moore2002fast}, its license allows corpora produced by it to be freely distributed. 
It is also well documented, was easy to compile on my system\footnote{MacBook Air, M1 2020, 8GB RAM, running MacOS Monterey 12.3.1} and runs fast (aligning around \numprint{100000} sentences takes about three minutes). 

I tried, just for the sake of interest, to use Vecalign on a small portion of my corpus (300 sentences). 
Veclaign requires that all adjacent sentences be concatenated first (to allow for 1-to-many alignments). 
Then for each sentence-concatenation, the sentence embeddings have to be obtained from the LASER language model. 
Only then, can sentence alignment be calculated \autocite{thompson-koehn-2019-vecalign}. 

The process of obtaining the sentence alignment took quite some time---around 10 minutes for 300 sentences---and by quick inspection with the bare eye, the result wasn't better than the one achieved with hunalign, but rather worse. 
Obviously, this may be due to the fact that Romansh is not one of the languages LASER was trained on. 
That being said, LASER \emph{has} been said to generalize to unseen languages that are similar to the ones the model was trained on, e.g., Swiss German or West Frisian, which are similar to German and Dutch, respectively \autocite{artexte-schwenk-2019-laser} 
\footnote{See also \url{https://github.com/facebookresearch/LASER}}.

Since the corpus at hand is well-structured---the documents are pre-aligned, the translations are close translations, paragraphs in the source language correspond to paragraphs in the target language and the press releases are usually not longer than a few sentences---hunalign performed excellently. 
I didn't create a gold standard for sentence alignment, so automatic evaluation was not possible, but during the task of annotating word alignments for the gold standard of German-Romansh (see Chapter~\ref{chap:gold-standard}), I only had to discard 11 out of 611 sentences due to  misalignment. 
This corresponds to a precision of 98.2\% or an error rate of 1.8\%.

\subsection{Pipeline}

\begin{figure}
\centering
\scalebox{0.9}{
\begin{tikzpicture}[node distance=2cm]
	\node (a) [startstop, align=center] {Query Documents \\From Database};
	\node (b) [startstop, right of=a, xshift=2cm] {Sentence Segmentation};
	\node (c) [startstop, right of=b, xshift=2cm, align=center] {Align \\ Language Pairs};
	\node (d) [startstop, right of=c, xshift=2cm] {Filter and Tokenize};
	\draw [arrow] (a) -- (b);
	\draw [arrow] (b) -- (c);
	\draw [arrow] (c) -- (d);
\end{tikzpicture}}
\caption{Sentence alignment pipeline}
\label{fig:sentence-alignment}
\end{figure}

The scripts responsible for compiling the sentence pairs are under the folder \texttt{align\_sentences} on my repository on GitHub. 
The bash script \texttt{make\_bicorpus.py} is responsible for executing the pipeline.

Figure~\ref{fig:sentence-alignment} visualizes the steps taken for sentence alignment.

\subsection{Database Query and Sentence Segmentation}
In the first step, all aligned documents are extracted from the corpus and are written to monolingual files, one sentence per line, and one file per year.
This is done by querying the SQLite database for all the aligned documents for each year, a task for which the script \texttt{exctract\_multicorpus.py} is responsible.

Sentence segmentation (also called sentence tokenization) was done using NLTK's Punkt tokenizers \autocite{bird-2009-nltk}. 
Since I wasn't able to integrate a sentence tokenizer for Romansh into the pipeline, I used the an NLTK Punkt tokenizer model which was trained on Italian. 
After instantiating both the German and the Italian models, I extended the list of abbreviations\footnotemark~to enhance the performance of the tokenizer and avoid wrong segmentation. 

\footnotetext{The abbreviations for Romansh were kindly taken from Lisa Gasner's/Samuel Läubli's GitHub repository.}

In the course of sentence segmentation, paragraphs are retained by converting line breaks into  special \texttt{<p>} tokens. 
These tokens will serve hunalign as anchor points for sentence alignment, cf.,  Section~\ref{sec:hybrid-models}.

The result is three files for each year, one for each language, containing one sentence per line and \texttt{<p>} tokens marking paragraph borders. 
Further, to keep the corpus well-structured, the file ID (cf., section~\ref{subsubsec:web-scraper}) is included at the beginning of each document. 
In case there is no mutual file ID, the date is included. 
The file IDs/dates will be used by \texttt{hunalign} as anchor points for keeping the documents aligned, see Listing~\ref{lst:sentences} for an example.

\vspace{0.25cm}

\begin{lstlisting}[language=txt, caption={[Excerpt from a file containing sentences for alignment.]Excerpt from a  file containing sentences for alignment. 
In order to keep the file structured and increase alignment performance, each document starts with a date and paragraph are boundaries are marked with a special \texttt{<p>} token.}, captionpos=t, label={lst:sentences}]
2004-01-27
www.gr.ch neu mit Online-Schalter und mit Interessenbindungen des Grossen Rats
Ein neues, zentrales Element von www.gr.ch ist der integrierte Behörden-Online-Schalter www.ch.ch.
...
Der Online-Schalter wird laufend in Zusammenarbeit zwischen Bund, Kantonen und Gemeinden weiterentwickelt und inhaltlich erweitert.
<p>
Parlament: Interessenbindungen öffentlich einsehbar 
...
Weiter wurden die Funktionalitäten der Stichwortsuche verbessert, der Informationsgehalt im Bereich "Unser Kanton" erweitert ("Produkte aus Graubünden", Suchmaschine für Graubünden) sowie der Sprachenwechsel zwischen den Inhalten in deutsch, romanisch und italienisch vereinfacht.
<p>
Standeskanzlei: Leitbild neu im Internet 
...
Zudem verrät www.staka.gr.ch auch, warum ein Picasso und der Begriff "Light" ohne weiteres mit der Standeskanzlei Graubünden in Zusammenhang gebracht werden können.
<p>
 Die neuen Web-Inhalte finden Sie hier: 
 - Online- Schalter 
 - Mitglieder 
 - Stellvertreter
 - www.staka.gr.ch 
<p>
 Gremium: Standeskanzlei Graubünden 
 Quelle: dt Standeskanzlei Graubünden
\end{lstlisting}

\subsection{Aligning Language Pairs}
\label{sec:aligning-language-pairs}
As described in Section~\ref{subsec:tool}, my tool of choice for aligning the sentence is hunalign. 
hunalign can use a bilingual dictionary for alignment, but the existence of such a dictionary is not a real restriction. 
In the absence of such a dictionary, the program will first fall back to sentence-length information, then automatically build a dictionary based on this alignment, and finally use this automatically-built dictionary for alignment in a second pass\footnote{\url{https://github.com/danielvarga/hunalign}}.

Although inspection with the bare eye revealed excellent precision (from the 611 sentences extracted for annotation of word alignment for the gold standard, only 11 were misalignments) which means the absence of a pre-made dictionary is not obstacle, when aligning the entire corpus, I used the German--Rumantsch Grischun dictionary downloaded from the online dictionary \emph{Pledari Grond}\footnote{\url{https://www.pledarigrond.ch/rumantschgrischun}} to support hunalign even further.

Files for three language pairs are  then created: German--Romansh, German--Italian and Romansh--Italian, one file for each year. 
The files for each language combination are then concatenated.
The result is three files containing all the sentence pairs for each language combination, from 1997 until today.

\subsection{Filtering and Tokenizing}
The press releases often contain sentences that are repeated throughout many of them, such as noting the source of the information at the end of the press release. 
A very common sentence ending a press release in German is \emph{Quelle: dt Standeskanzlei Graubünden} \enquote{Source: German State Chancellory Grisons}. 
Such duplicate sentences are not simply redundant in the corpus, but might also be considered noise in the data. % which might negatively influence an \acrshort{nmt} model trained on this corpus \autocite{thompson-koehn-2019-vecalign}. 
Misaligned sentences and untranslated sentences are also considered noise that can have a negative influence on \acrshort{nmt} models \autocite{khayrallah-koehn-2018-impact}.
Therefore, duplicates and untranslated sentences should be filtered out, in order to make sure the remaining pairs are of high quality.

The script \texttt{filter\_bicorpus.py} takes a file generated by hunalign (containing three tab-seperated columns: source--target--score) and produces a tab-separated file containing two columns (source and target) with the filtered sentences, one sentence per line and word-tokenized. 
The script removes sentences containing E-Mails, \acrshort{URL}s or phone numbers, as well as sentences where source and target languages are identical, i.e., untranslated sentences. Sentences in which the difference in character length  between source and target is too large (more than three times), for which I then assume misalignment, are also removed.

Word tokenization is important for the next step---word alignment. 
For the task of tokenization, I used NLTK's \autocite{bird-2009-nltk} word tokeniziation functions, while applying the German model for German text and the Italian model for Romansh and Italian text. 
The justification for the latter is that Romansh, in a manner very similar to Italian, uses apostrophes to attach enclitics (articles and pronouns) to neighboring words, which should be separated for word tokenization. 
An inspection with the bare eye looked precise enough. 
In the course of annotating the word alignment for the gold standard, I had to correct the tokenization less than 10 times for 600 sentences.


\section{Results}
The resulting final parallel corpus consists of three files containing around \numprint{80000} unique sentence pairs for each of the three language combinations: German--Romansh, German--Italian and Romansh--Italian. 
Each line in the file contains a sentence pair, separated by a tab character (see Listing~\ref{lst:bicorpus}).

Table~\ref{tab:bicorpus-stats} elaborates on the number of sentences, tokens and type for each combination. 


\begin{table}[h]
\centering
\begin{tabular}{lp{1.5cm}p{2cm}p{1.5cm}p{2cm}p{1.5cm}}
\toprule 
Combination    & Sentence pairs        &  Tokens Source      &  Types Source     & Tokens \mbox{Target} & Types \mbox{Target} \\
\midrule 
German--Romansh & \numprint{79613} & \numprint{1400313} & \numprint{80239} & \numprint{1792851} & \numprint{42656} \\

German--Italian & \numprint{78186} &  \numprint{1396933} &   \numprint{80149} & \numprint{1685792}&  \numprint{48854} \\ 

Romansh--Italian & \numprint{78101} & \numprint{1760424} & \numprint{42295} &  \numprint{1655822} & \numprint{48753} \\
\bottomrule
\end{tabular}
\caption[Parallel corpus in numbers]{Parallel corpus in numbers, as of July 20, 2022. 
\enquote{Sentences} are sentence pairs.
\enquote{Source} refers to the language on the left and \enquote{Target} to the language on the right, not necessarily to the actual source language of the translation.}
\label{tab:bicorpus-stats}
\end{table}

\vspace*{1cm}

\begin{lstlisting}[language=txt, caption={Excerpt from the file containing sentence pairs in German--Romansh}, captionpos=b, label={lst:bicorpus}]
Das kantonale Personal und die Volksschullehrerinnen und -lehrer müssen auf einen Teuerungsausgleich verzichten .	Il persunal chantunal e las scolastas ed ils scolasts da las scolas popularas ston desister d' ina gulivaziun da la chareschia .
Mit diesem Lohnopfer leisten sie in Würdigung der angespannten Finanzlage des Kantons und der schwachen Wirtschaftslage einen Beitrag dazu , die Kosten einzudämmen .	Cun quest sacrifizi da salari prestan els , a vista da la situaziun precara da las finanzas chantunalas e da la flaivla economia , ina contribuziun per franar ils custs .
Die Teilrevision des Behindertengesetzes wird auf Anfang 1998 in Kraft gesetzt .	La revisiun parziala da la lescha dals impedids vegn messa en vigur cun l' entschatta da 1998 

\end{lstlisting}


