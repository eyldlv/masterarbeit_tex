\chapter{Gold standard}
In order to measure the quality of words alignments, a model's performance is measured on a test set which is a gold standard created by human annotators. 
For the gold standard to be of good quality and consistent with itself, annotators have to follow strict guidelines.
These guidelines address issues of ambiguity in word alignments. \autocite[115]{koehn2009}. 

Some problematic cases that might occur are function words (TODO) that have no clear equivalent in the other language. 
\cite{koehn2009} gives as an example the German-English sentence pair: \emph{John wohnt hier nicht}  \emph{John does not live here}. 
What German word should the English word \emph{does} be aligned to? 
Three different choices can be made:
\begin{enumerate}
	\item The word should remain unaligned since it has to clear equivalent in German.
	\item The word \emph{does} is connected with \emph{live}; it contains the number and tense information which is in German contained in one word \emph{wohnt}, so it should be aligned to \emph{wohnt}, together with \emph{live}.
	\item \emph{does} is part of the negation; without it, the sentence would not contain this word. Therefore, \emph{does} should be aligned with \emph{nicht} (the German negation).

\section{Sure and Possible alignments}
An approach for solving problematic cases is the distinction between \emph{sure} (s) and \emph{possible} (p) alignments \autocite[115]{koehn2009}, which are also sometimes referred as fuzzy alignments \autocite{clematide2018}. 
Generally, these labels allow to distinguish between ambiguous and unambiguous links. Ambiguous links are labeled \emph{possible} and unambiguous links are labeled \emph{sure} \autocite{lambert2005}. This distinction also has an impoact on the way the evaluation metrics are computed (more on that later).

There seems to be no clear global definition about which alignments should be considered as umabiguous and marked as \emph{sure} and which should be considered ambiguous marked as \emph{possible}. 
For some created gold standards, no distinciton between \emph{sure} and \emph{possible} alignments was made \autocite{clematide2018}. 
In another case, annotators were asked to first label all alignments as \emph{sure} and then refine their alignments with confidence labels \autocite{holmqvist-ahrenberg-2011-gold}. 
In the creation of the English-Icelandic gold standard in \cite{steingrimsson-etal-2021-combalign},  annotators used only \emph{sure} links. 
Their annotations were then combined, with all 1-1 alignments both annotators agreed upon (i.e. the intersection of their annotations) makred as \emph{sure} and differences all other alignments made by either one or both were marked as \emph{possible} \autocite{steingrimsson-etal-2021-combalign}. 


\end{enumerate}

\section{Guidelines}